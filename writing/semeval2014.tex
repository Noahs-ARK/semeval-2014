%
% File twocolumn.tex
%
%%
%% Based on the style files for *SEM-2014, which were, in turn,
%% Based on the style files for COLING-2014, which were, in turn,
%% Based on the style files for ACL-2014, which were, in turn,
%% Based on the style files for ACL-2013, which were, in turn,
%% Based on the style files for ACL-2012, which were, in turn,
%% based on the style files for ACL-2011, which were, in turn,
%% based on the style files for ACL-2010, which were, in turn,
%% based on the style files for ACL-IJCNLP-2009, which were, in turn,
%% based on the style files for EACL-2009 and IJCNLP-2008...

%% Based on the style files for EACL 2006 by
%%e.agirre@ehu.es or Sergi.Balari@uab.es
%% and that of ACL 08 by Joakim Nivre and Noah Smith

\documentclass[11pt]{article}
\usepackage{semeval2014}
\usepackage{times}
\usepackage{url}
\usepackage{latexsym}

\usepackage{enumitem}
\setitemize{noitemsep,topsep=10pt,parsep=0pt,partopsep=0pt}
\setenumerate{noitemsep,topsep=10pt,parsep=0pt,partopsep=0pt}

%\setlength\titlebox{5cm}

% You can expand the titlebox if you need extra space
% to show all the authors. Please do not make the titlebox
% smaller than 5cm (the original size); we will check this
% in the camera-ready version and ask you to change it back.

\newcommand{\wsname}{SemEval-2014}
\newcommand{\submissionpage}{\url{http://alt.qcri.org/semeval2014/index.php?id=cfp}}
\newcommand{\filename}{semeval2014}
\newcommand{\contact}{pnakov qf.org.qa}

\usepackage{amsmath}
\usepackage{amssymb}
\usepackage{graphicx}
\usepackage[usenames,dvipsnames]{color}
\definecolor{myblue}{rgb}{0,0.1,0.6}
\definecolor{mygreen}{rgb}{0,0.3,0.1}
\usepackage[colorlinks=true,linkcolor=black,citecolor=mygreen,urlcolor=myblue]{hyperref}
\newcommand{\transpose}{^\mathsf{T}}

\newcommand{\bocomment}[1]{\textcolor{Bittersweet}{[#1 -BTO]}}

\title{SemEval-2014 Task 8: Broad-Coverage Semantic Dependency Parsing}

\author{First Author \\
  Affiliation / Address line 1 \\
  Affiliation / Address line 2 \\
  Affiliation / Address line 3 \\
  {\tt email@domain} \\\And
  Second Author \\
  Affiliation / Address line 1 \\
  Affiliation / Address line 2 \\
  Affiliation / Address line 3 \\
  {\tt email@domain} \\}

\date{}

\begin{document}
\maketitle
\begin{abstract}
  abstract
\end{abstract}



\section{Introduction}



\section{Formalisms}

\begin{itemize}
\item Quick overview of three formalisms, emphasizing differences between them, consequences for parsing
\end{itemize}


\section{Models}
\newcommand{\logitedge}{\textsc{LogitEdge}}

We used the same set of (first-order) features
in two different models: a faster-to-train
logistic regression model (\logitedge, \ref{s:logitedge}),
and a structured graph prediction model (\ref{s:graphparser}).


\subsection{\logitedge parser} \label{s:logitedge}

This model treats every pair of words as a multiclass logistic regression,
either to be an edge with one of the formalism's labels, or a null
\textsc{NoEdge} decision.  


The model considers all token pairs $(i,j)$ within distance 10 of each other. \bocomment{TODO insert oracle recall note}  Both directions are considered.  If the gold standard has an edge in the direction $i \rightarrow j$, the direction $j\rightarrow i$ is considered a \textsc{NoEdge}.

Feature extractors are defined to extract features both for individual tokens (graph nodes), and for pairs of tokens, which are conjoined against all possible output labels.
For candidate head $i$ and child $j$, the model defines a distribution over
$y_{ij} \in \{\text{edge labels}\} \cup \{\textsc{NoEdge}\}$, using weights $\beta$,

\begin{align*} 
  P(y_{ij}=k; \beta) & \propto 
  \exp \big( \\
  &
  \beta^{\text{(node as head)}}_k \cdot f^{(node)}(i)
  \ \ + \\ 
  &
  \beta^{\text{(node as child)}}_k \cdot f^{(node)}(j)
  \ \ + \\
  &
  \beta^{\text{(pair)}}_k \cdot f^{(pair)}(i,j)
    \big)
\end{align*}

Training minimizes total negative log-likelihood of the above,
plus L2 regularization.  Adagrad is used for optimization (which seemed to achieve better solutions than L-BFGS, at least for earlier iterations).


\subsection{Graph-based parser} \label{s:graphparser}


\section{Features}

\subsection{all the stuff in our model}

\begin{itemize}
\item BasicFeatures
\item LinearOrderFeatures
\item CoarseDependencyFeatures
\item LinearContextFeatures
\item DependencyPathv1
\item SubcatSequenceFE
\item UnlabeledDepFE
\item Brown clusters
\end{itemize}
\subsection{feature hashing}

\subsection{all the stuff we tried but didn't work}


\section{Evaluation}

\section{Conclusion}

\nocite{flanigan-etal:ACL2014}



\section*{Acknowledgements}

\bibliographystyle{acl}
\bibliography{semeval8}




\end{document}
